\documentclass[12pt,a4paper]{article}

\usepackage[brazil]{babel}
\usepackage[utf8]{inputenc}
\usepackage{graphicx}

\title{Instruções de Uso do Componente PDF para Prisma}
\author{Douglas Rodrigues de Almeida}
\date{04 de Agosto de 2018}

\begin{document}

\maketitle

\section{Introdução}
O Componente PDF para Prisma é uma ferramenta para auxiliar a geração de documentos eletrônicos em PDF a partir da solicitação de impressão realizada no sistema Prisma.

\section{Configurando o Prisma}
Após a instalação do componente, é necessário configurar o Prisma para gerar arquivos em PDF. Essa configuração é individual, sendo registrada na matrícula do servidor. Portanto, só deve ser feita uma única vez, ainda que o usuário mude de computador.

\textbf{Para configurar sua matrícula no Prisma para gerar PDF:}

\begin{enumerate}
  \item Abra o Prisma e autentique-se com sua matrícula e senha.
  \item No menu principal, vá para BENEFÍCIOS.
  \item Pressione....
\end{enumerate}

\end{document}